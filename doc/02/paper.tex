\documentclass[a4paper,article,11pt,oneside]{article}
\usepackage{fullpage}

\begin{document}
\title{Report for OS Project \#2 (2011 Fall)\\Team 071}
\author{Jeehoon Kang\\\texttt{jhkang@ropas.snu.ac.kr}\and
  Jonghyuk Lee\\\texttt{jonghyuk0605@gmail.com} \and
  Taekwon Lee\\\texttt{tkwonlee@hotmail.com} }
\maketitle

This article is structured as follows. Section~\ref{secarg} deals with
how to pass arguments to user programs. Setion~\ref{secsys} presents
the implementation details of system calls, including process
management system calls, such as process halt, exit, execute, or
wait, and file management system calls. Section~\ref{secconclusion}
concludes.

Note that our Mercurial repository is based at
\texttt{https://bitbucket.org/lunaticas/os}. Later, we address
revisions or files without explicitly mentioning the repository.

We proudly report that we pass all 76 test cases out of 76.

\section{Argument Passing}\label{secarg}
When a user program starts, the program should get the arguments of
the command. For example, typing \texttt{echo x} in a shell should
execute program \texttt{echo} and pass \texttt{[``echo'', ``x'']} as
its arguments. The entry point of a typical C program is \texttt{main}
function, whose declaration is \texttt{int main (int argc, char
  *argv[])} where \texttt{argc} is the number of arguments and
\texttt{argv} is the array of argument strings. Function
\texttt{argument\_parse} parses the argument.
\begin{verbatim}
(in process.c)
static void
argument_parse (char *cmd, int *argc, char ***argv);
\end{verbatim}

As 80x86 calling convention, we have to pass arguments to the stack
before the user program starts. We parse in \texttt{process\_execute}
and pass it to \texttt{start\_process} because
\texttt{process\_execute} has to deal with error cases and
\texttt{start\_process} only can access to \texttt{if\_.esp}. We
assume that the total length of the argument does not exceed
\texttt{PGSIZE / 2}.

\begin{verbatim}
(in process.c, process_execute)
  strlcpy (cmd, cmd_, PGSIZE);
  argument_parse (cmd, argc, &argv);
  *((char ***)(cmd + PGSIZE / 2 + sizeof (int))) = argv;
(in process.c, start_process)
  argc = *((int *)(cmd + PGSIZE / 2));
  argv = *((char ***)(cmd + PGSIZE / 2 + sizeof (int)));
\end{verbatim}

We put the arguments in the reverse order to the stack
\texttt{if\_.esp}. After this, we put pointers to the argument strings
in the revert order, the pointer to the array of pointers to the
argument strings, and the fake return address \texttt{NULL}.
\begin{verbatim}
(in process.c, start_process)
  for (i = argc - 1; i >= 0; --i)
  {
    len = strlen (argv[i]);
    if_.esp -= (len + 1) * sizeof (char);
    strlcpy (if_.esp, argv[i], len + 1);
    argv[i] = if_.esp;
  }

  if_.esp -= ((size_t) if_.esp) % sizeof (size_t);
  if_.esp -= sizeof (char *);
  *((char **) if_.esp) = NULL;

  for (i = argc; i >= 0; --i)
  {
    if_.esp -= sizeof (char *);
    *((char **) if_.esp) = argv[i];
  }

  if_.esp -= sizeof (void *);
  *((void **) if_.esp) = if_.esp + sizeof (size_t);

  if_.esp -= sizeof (size_t);
  *((size_t *) if_.esp) = argc;

  if_.esp -= sizeof (void *);
  *((void **) if_.esp) = NULL;
\end{verbatim}

Then we set the thread name to be the first argument, i.e. the name of
program, and free allocated pages for arguments.
\begin{verbatim}
(in process.c, start_process)
  strlcpy (thread_current ()->name, argv[0], sizeof (thread_current ()->name));
  palloc_free_page (cmd);
  palloc_free_page (argv);
\end{verbatim}

By allocating a new page for pointers to argument strings, it scales
well unless the total length of command is over \texttt{PGSIZE / 2}.

However, only static number of characters of the name of programs
are stored in \texttt{struct thread} and printed when processes
terminate. We decided this is not a serious problem and if it is, we
can allocate a page for the name.

\section{System Calls}\label{secsys}
\subsection{User Memory Access}
System call is the channel between user programs and the kernel so
that handlers need to read from and write to the user memory. The
Pintos documentation provides a way to to this by magic assembly
code. We abstract these details to 1-byte read/write function and
4-byte read/write function. Note that 80x86 is little endian :)
\begin{verbatim}
(in syscall.c)
static int get_user (const uint8_t *uaddr) { ... }
static bool put_user (uint8_t *udst, uint8_t byte) { ... }
static int get_user_4 (const uint8_t *uaddr) { ... }
static bool put_user_4 (uint8_t *udst, int value) { ... }
\end{verbatim}

Every system call looks 1 to 4 4-byte values from the stack and
returns a 4-byte value. The handler dispatches system calls by its
first 4-byte value input and determines the number of additional
input values.
\begin{verbatim}
(in syscall.c)
#define SYSCALL(esp) (get_user_4 (esp))
#define ARG0(esp) (get_user_4 ((esp) + 4))
[...]

static void
syscall_handler (struct intr_frame *f) 
{
  int result;
  switch (SYSCALL (f->esp))
  {
  case SYS_HALT: result = handle_halt (); break;
  case SYS_EXIT: result = handle_exit (ARG0(f->esp)); break;
  [...]
  }

  put_user_4 (f->eax, result);
}
\end{verbatim}

\subsection{Process Management System Calls}\label{secprocess}
We have to implement 4 process management system calls: \texttt{halt,
  exit, exec, wait}. \texttt{halt} and \texttt{wait} are relatively
easy; they just call other functions. An appropriate lock for files is
needed to prevent illegal access.

\begin{verbatim}
(in process.c)
static int
handle_halt (void)
{
  thread_exit ();
}

static int
handle_exec (const char *cmd)
{
  lock_acquire (&file_lock);
  int result = process_execute (cmd);
  lock_release (&file_lock);
  return result;
}
\end{verbatim}

\texttt{exit} and \texttt{wait} have to cooperate with each other so
that \texttt{wait} can receive the exit code from \texttt{exit}. Our
design relies on \texttt{process.c} and \texttt{thread.c} for this
cooperation and \texttt{syscall.c} does nearly nothing. \texttt{exit}
just set the exit code to \texttt{struct thread} and print
end-of-program message.

\begin{verbatim}
(in process.c)
static int
handle_exit (int exit_code)
{
  thread_current ()->exit_code = exit_code;
  printf ("%s: exit(%d)\n", thread_current ()->name, exit_code);
  thread_exit ();
}

static int
handle_wait (pid_t pid)
{
  return process_wait (pid);
}
\end{verbatim}

To implement these functionalities, we store more data on
\texttt{struct thread}. \texttt{exit\_code} stores the exit code of the
thread, \texttt{wait\_sema} provides synchronization for
\texttt{process\_wait}, \texttt{load\_sema} and \texttt{load\_status}
for \texttt{process\_execute} to know whether the load succeed,
and \texttt{parent} for storing who is the parent.
\begin{verbatim}
(in thread.h, struct thread)
#ifdef USERPROG
    int exit_code;
    struct semaphore wait_sema;
    struct semaphore load_sema;
    struct thread *parent;
    struct file *self;
    bool load_status;
#endif
\end{verbatim}

When a user program is loaded, \texttt{start\_process} signals whether
the load was successful to \texttt{process\_execute} using a semaphore.
\begin{verbatim}
(in process.c, process_execute)
  tid = thread_create (cmd, PRI_DEFAULT, start_process, cmd);
  if (tid == TID_ERROR)
  {
    palloc_free_page (cmd);
    return TID_ERROR;
  }

  struct thread *t = thread_lookup (tid);
  sema_down (&t->load_sema);

  if (!t->load_status)
  {
    return TID_ERROR;
  }

  return tid;

(in process.c, start_process)
  if (!success)
  {
    palloc_free_page (cmd);
    thread_current ()->load_status = false;
    sema_up (&thread_current ()->load_sema);
    handle_exit (-1);
  }

  (...)

  palloc_free_page (cmd);
  thread_current ()->load_status = true;
  sema_up (&thread_current ()->load_sema);
\end{verbatim}

Originally, when \texttt{thread\_exit} is called, the page for the
thread is freed immediately. However, to satisfy the specification for
\texttt{wait}, we have to remember the exit codes of already dead
threads. So instead of freeing immediately after a thread is exited,
we add the thread to the \texttt{dead\_list} and \texttt{sema\_up} the
semaphore \texttt{wait\_sema} for \texttt{process\_wait} to get a
signal to wake up.
\begin{verbatim}
(in thread.c, thread_exit)
  list_push_back (&dead_list, &t->allelem);
  sema_up (&t->wait_sema);
(in thread.c, schedule_tail, when a thread is dying)
#ifndef USERPROG
      palloc_free_page (prev);
#endif
\end{verbatim}

If a thread is terminated, the exit codes of the children of it are no
more needed. In \texttt{thread\_exit}, we free the memory for this information.
\begin{verbatim}
(in thread.c, thread_exit)
  struct list_elem *e;
  for (e = list_begin (&dead_list); e != list_end (&dead_list); )
  {
    struct thread *t = list_entry (e, struct thread, allelem);
    e = list_next (e);
    if (t->parent == thread_current ())
    {
      thread_remove (t);
    }
  }    
\end{verbatim}

When a thread waits for one of its child, now the thread just
\texttt{sema\_down} the semaphore \texttt{wait\_sema} of the child,
which is acquired only after the child thread die. After the parent
getthe exit code of the child, the child is removed completely because
its exit code is no more needed.
\begin{verbatim}
(in process.c, process_wait)
int
process_wait (tid_t child_tid) 
{
  int result;
  struct thread *child = thread_lookup (child_tid);

  if (child == NULL || child->parent != thread_current ())
  {
    return -1;
  }

  sema_down (&child->wait_sema);
  result = child->exit_code;
  thread_remove (child);
  return result;
}
\end{verbatim}

For this, we have to implement \texttt{thread\_lookup} which finds
the thread of given \texttt{tid} regardless of their liveness. To list
up dead threads, we defined \texttt{dead\_list}. \texttt{thread\_remove}
removes a thread from \texttt{dead\_list} and free the memory of
\texttt{struct thread}.

\begin{verbatim}
(in thread.c)
void
thread_remove (struct thread *t)
{
  enum intr_level old_level;    
  old_level = intr_disable ();
  list_remove (&t->allelem);
  palloc_free_page (t);
  intr_set_level (old_level);  
}
\end{verbatim}

In our design, \texttt{process\_wait} does not block for already dead
thread and with a semaphore, we gurantee it returns as soon as the
child die (and of course, after the parent is scheduled to run).

However, we does not free one page for each \texttt{struct thread}
whose parent did neither \texttt{process\_wait} nor
\texttt{thread\_exit} yet. We decided this is not a serious problem
and if it is, we can introduce a new data structure for efficiently
storing exit codes.

\subsection{File Management System Calls}\label{secfile}

We implement \texttt{create, remove, open, filesize, read, write,
  seek, tell} and \texttt{close}. In \texttt{fd.c}, we manage file
descriptors. When we handle files, we resort to functions in
\texttt{filesys.c, file.c}.

We introduce a lock named \texttt{file\_lock}, because file processing
should be mutually protected. When another thread handles the file, a
thread has to wait.

File descriptor is an abstract indicator for accessing a file. A
process can manage many files so that a process can have many file
descriptors. A file descriptor is a nonnegative integer global to the
system, so file descriptor manager has to remember the owner, the
file, and the status of each file descriptor.

\begin{verbatim}
(in fd.c)
struct file_descriptor
{
  struct file *file;
  struct thread *owner;
  bool status;
};

void
fd_init (void)
{
  fd_cnt = 3;
  empty_fd_cnt = 0;
}
\end{verbatim}

Note that the number 0 is for \texttt{STDIN}; 1, \texttt{STDOUT}; 2,
\texttt{STDERR}. File descriptor manager consists of auxiliary
functions including validity check, file descriptor allocation,
deallocation, and file access.

\begin{verbatim}
(in fd.c)
fd_is_valid (int fd_id)
{
  if (fd_id < 3 || fd_id >= fd_cnt)
  {
    return false
  }
  if (fd_list[fd_id].status == false)
  {
    return false
  }
  if (fd_list[fd_id].owner->tid != thread_current()->tid)
  {
    return false
  }
  return true
}

fd_acquire (struct file *file)
{
  int fd_id;
  if (empty_fd_cnt > 0)
  {
    fd_id = empty_fd_list[--empty_fd_cnt];
  }
  else
  {
    fd_id = fd_cnt++;
  }
  fd_list[fd_id].file = file;
  fd_list[fd_id].owner = thread_current ();
  fd_list[fd_id].status = true;
  return fd_id;
}

void fd_release (int fd_id)
{
  file_close (fd_list[fd_id].file);
  fd_list[fd_id].status = false;
  empty_fd_list[empty_fd_cnt++] = fd_id;
}

struct file *
fd_get (int fd_id)
{
  return fd_list[fd_id].file;
}
\end{verbatim}

When a process terminates, all the files assigned to the process
should be released. In \texttt{process\_exit},
\texttt{fd\_process\_exit} is called.

\begin{verbatim}
(in fd.c)
void
fd_process_exit (struct thread *t)
{
  int i;
  for (i = 3; i < fd_cnt; ++i)
  {
    if (fd_list[i].owner->tid == t->tid &&
        fd_list[i].status)
    {
      fd_release (i);
    }
  }
}
\end{verbatim}

With the file descriptor manager, many file-related system calls are
implemented in a straightforward and natural way.

\begin{verbatim}
(in syscall.c)
static int
handle_create (const char *file_name, off_t file_size)
{
  if (!is_valid_memory_filecheck (file_name))
  {
    handle_exit(-1);
  }
  lock_acquire (&file_lock);
  int result = filesys_create (file_name, file_size);
  lock_release (&file_lock);
  return result;
}

static int
handle_remove (const char *file_name)
{
  if (!is_valid_memory_filecheck (file_name))
  {
    handle_exit(-1);
  }
  lock_acquire (&file_lock);
  int result = filesys_remove (file_name);
  lock_release (&file_lock);
  return result;
}
\end{verbatim}

Handlers \texttt{handle\_create} and \texttt{handle\_remove} relies on
predefined \texttt{filesys\_create} and \texttt{filesys\_remove}. If
the address of file name is not valid, it is error so we exit and
return -1. We acquire the \texttt{file\_lock} before calling file
managing functions for avoiding collision.

\begin{verbatim}
(in syscall.c)
static int
handle_open (const char *file_name)
{
  if (!is_valid_memory_filecheck (file_name))
  {
    handle_exit(-1);
  }

  lock_acquire (&file_lock);
  struct file *file_opened = filesys_open (file_name);

  if (file_opened == NULL)
  {
    lock_release (&file_lock);
    return -1;
  }

  int result = fd_acquire (file_opened);
  lock_release (&file_lock);
  return result;
}
static int
handle_close (const int fd_id)
{
  if (!fd_is_valid (fd_id))
  {
    handle_exit (-1);
  }
  
  lock_acquire (&file_lock);
  fd_release (fd_id);
  lock_release (&file_lock);
  return 0;
}
\end{verbatim}

Handlers \texttt{handle\_open} and \texttt{handle\_close} relies on
predefined \texttt{filesys\_open} and \texttt{file\_close}. We also need to validate address or acquire lock in this function. And in \texttt{handle\_open} function, after we open file with \texttt{filesys\_open}, we call \texttt{fd\_acquire} to get file descriptor. Then \texttt{fd\_acquire} in \texttt{fd.c} returns valid file descriptor of this opened file. When we close file, we call \texttt{fd\_release} to release file descriptor. Closing file process is included in there.

\begin{verbatim}
(in syscall.c)
static int 
handle_read (const int fd_id, char *buf, const int size) 
{
  if (!is_valid_memory_filecheck (buf) || !is_user_vaddr (buf + size))
  {
    handle_exit (-1);
  }

  lock_acquire (&file_lock);
  if (fd_id == STDIN_FILENO)
  {
    int i;
    for (i = 0; i < size; ++i)
    {
      buf[i] = input_getc ();
    }
    lock_release (&file_lock);
    return size;
  }

  if (!fd_is_valid (fd_id))
  {
    lock_release (&file_lock);
    return -1;
  }

  struct file *fp = fd_get (fd_id);
  int result = file_read (fp, buf, size);

  lock_release (&file_lock);
  return result;
}
\end{verbatim}

Handler \texttt{handle\_read} relies on predefined
\texttt{input\_getc} and \texttt{file\_read}. After we validate
address of buffer, we determine whether it is standard input or
not. If it is standard input, we use input\_getc to get input. If it
is not, first we check whether file descriptor is valid or not by
calling  \texttt{is\_valid\_fd} in  \texttt{fd.c}. After it, we call
\texttt{file\_read} function to read file. This function is also
protected by the \texttt{file\_lock}.

\begin{verbatim}
(in syscall.c)
static int 
handle_write (const int fd_id, const char *buf, const int size)
{
  if (!is_valid_memory_filecheck (buf) || !is_user_vaddr (buf + size))
  {
    handle_exit (-1);
  }

  lock_acquire (&file_lock);
  if (fd_id == STDOUT_FILENO)
  {
    putbuf (buf, size);
    lock_release (&file_lock);
    return size;
  }
  
  if (!fd_is_valid (fd_id))
  {
    lock_release (&file_lock);
    return -1;
  }

  int result = file_write (fd_get (fd_id), buf, size);
  lock_release (&file_lock);
  return result;
}
\end{verbatim}

Handler \texttt{handle\_write} relines on predefined \texttt{putbuf}
and  \texttt{file\_write}. After we validate address of buffer, we
determine whether it is standard output or not. If it is standard
output, we use putbuf to put output. If it is not, we check whether
file descriptor is valid or not here too. After it, we call
\texttt{file\_write} function to write file. This function is also
protected by the \texttt{file\_lock}.

\begin{verbatim}
(in syscall.c)
static int
handle_seek (const int fd_id, const int position) 
{
  if (!fd_is_valid (fd_id))
  {
    return -1;
  }

  lock_acquire (&file_lock);
  file_seek (fd_get (fd_id), position);
  lock_release (&file_lock);
  return 0;
}

static int
handle_tell (const int fd_id) 
{
  if (!fd_is_valid (fd_id))
  {
    return -1;
  }

  lock_acquire (&file_lock);
  int result = file_tell (fd_get (fd_id));
  lock_release (&file_lock);
  return result;
}
\end{verbatim}

Handlers \texttt{handle\_seek} and  \texttt{handle\_tell} relies on
predefined \texttt{file\_seek} and  \texttt{file\_tell}. We also need
to validate address or acquire lock in this function. And in both
functions, we find the file of given file descriptor by calling \texttt{fd\_get} of \texttt{fd.c}.

\begin{verbatim}
(in syscall.c)
static int
handle_filesize (const int fd_id) 
{
  if (!fd_is_valid (fd_id))
  {
    handle_exit(-1);
  }
   
  lock_acquire (&file_lock);
  int result = file_length (fd_get (fd_id));
  lock_release (&file_lock);
  return result;
}
\end{verbatim}

Handler \texttt{handle\_filesize} relies on predefined
\texttt{file\_length}. We also need to validate address or acquire lock in this function. And in this function, we find file with given file descriptor by calling  \texttt{fd\_get} function in \texttt{fd.c}.

\subsection{Page Faults}\label{secpagefault}
When we meet a real page fault, we have to exit with code -1. So we modified
it.

\texttt{exception.c}.

\begin{verbatim}
(in exception.c, page\_fault)
  if (not_present || (is_kernel_vaddr (fault_addr) && user))
  {
    handle_exit(-1);
  }
\end{verbatim}

We exit with code -1 when there exists page fault and we also changed some \texttt{thread\_exit} in kill function to \texttt{handle\_exit(-1)}.

In addition, user program must not read kernel data. So we added some code in process loading function.

\begin{verbatim}
(in process.c, load)
        case PT_LOAD:
          if (phdr.p_vaddr < PGSIZE) break;
\end{verbatim}

This prevents user program to read from kernel data.

\section{Denying writes to Executables}\label{secdeny}
We have to protect executable files when they are running. So each process remember its own file name and denies write when they are running.

\begin{verbatim}
(in thread.h, struct thread)
    struct file *self;
\end{verbatim}

\begin{verbatim}
(in process.c, start_process)
  thread_current ()->self = filesys_open( argv[0] );
  file_deny_write(thread_current() -> self);
\end{verbatim}

\texttt{self} is file of thread's executable. When we start process, we call \texttt{file\_deny\_write} function to deny write of that file.

\begin{verbatim}
(in process.c, process_exit)
  file_close ( cur->self );
  cur -> self = NULL;
\end{verbatim}

When process ends, we close executable and initiate.

\section{Conclusion}\label{secconclusion}
We implemented argument passing, process management system calls, file
management system calls, and basic executable write protection. We
pass all 76 test cases out of 76 test cases.

\end{document}
