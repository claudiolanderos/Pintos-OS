\documentclass[a4paper,article,11pt,oneside]{article}
\usepackage{fullpage}

\begin{document}
\title{Report for OS Project \#3 (2011 Fall)\\Team 071}
\author{Jeehoon Kang\\\texttt{jhkang@ropas.snu.ac.kr}\and
  Jonghyuk Lee\\\texttt{jonghyuk0605@gmail.com} \and
  Taekwon Lee\\\texttt{tkwonlee@hotmail.com} }
\maketitle

\section{Introduction}
For this projec,t we changed \texttt{page.[ch], frame.[ch], swap.[ch]}
in \texttt{vm} directory and \texttt{init.[ch], thread.[ch],
  exception.[ch], process.[ch]} in \texttt{threads, userprog}
directories.

This article is structured as follows. Section~\ref{sechandler} shows
how default page fault handler is changed. Section~\ref{sectables}
shows how page and frame tables are used, while section~\ref{secimpl}
shows how to implement them. Section~\ref{secswap} shows how to swap
in and out frames. Section~\ref{secconclusion} concludes.

Note that our Mercurial repository is based at
\texttt{https://bitbucket.org/lunaticas/os}. Later, we address
revisions or files without explicitly mentioning the repository.

We report that we pass all required test cases in project 3 except for \texttt{pt-write-code}.

\section{Page Fault Handler}\label{sechandler}
The page fault handler shipped with brand-new pintos distribution
aborts the machine. To implement virtual memory, we changed the
behavior of the page fault handler as follows.

\begin{verbatim}
- in page_fault, exception.c
if (check_fault (f, fault_addr, not_present, user)) {
  handle_exit (-1);
}
\end{verbatim}

\section{Page and Frame Tables}\label{sectables}
To implement virtual memory, we have to maintain a page table for each
thread and the global frame table.
\begin{itemize}
\item In booting procedure, init the frame.
\begin{verbatim}
- in main, init.c
  init_frame();
\end{verbatim}
\item Each thread has its own page table. When a thread begins and
  ends, it initializes and free the page and frame tables.
\begin{verbatim}
- in struct thread, thread.h
  struct hash pagetable;
- in thread_create, thread.c
  page_table_init (&t->pagetable);
- in thread_exit, thread.c
  lock_acquire (&lock_ft);
  frame_table_free (thread_current ());
  page_table_free (thread_current ());
  lock_release (&lock_ft);
\end{verbatim}
\item When loading a user process, we allows
  \texttt{palloc\_get\_page} to fail. When it is the case, we victim a
  existing page and use it.
\begin{verbatim}
- in load_segment, process.c
    if (kpage == NULL)
    {
      kpage = victim ();
    }
    ASSERT (kpage != NULL);
\end{verbatim}
After loading a page, we set the page table entry for it.
\begin{verbatim}
- in load_segment, process.c
    /* Load this page. */
    lock_release (&lock_ft);
    if (file_read (file, kpage, page_read_bytes) != (int) page_read_bytes)
    {
      palloc_free_page (kpage);
      return false; 
    }
    lock_acquire (&lock_ft);
    memset (kpage + page_read_bytes, 0, page_zero_bytes);
    /* Add the page to the process's address space. */
    if (!install_page (upage, kpage, writable)) 
    {
      palloc_free_page (kpage);
      lock_release (&lock_ft);
      return false; 
    }
    make_frame_table ( upage, t);
    page_set_entry ( upage, writable, kpage);
    lock_release (&lock_ft);
 /* Advance. */
    read_bytes -= page_read_bytes;
    zero_bytes -= page_zero_bytes;
    upage += PGSIZE;
    off += page_read_bytes;
\end{verbatim}
\item Similary for stacks.
\begin{verbatim}
- in setup_stack, process.c
  if (kpage == NULL)
  {
    kpage = victim ();
  }
- in setup_stack, process.c
  success = install_page (((uint8_t *) PHYS_BASE) - PGSIZE, kpage, true);
  if (success)
  {
    *esp = PHYS_BASE;
    uint8_t *upage = ((uint8_t *) PHYS_BASE) - PGSIZE;
    make_frame_table (upage, thread_current());
    page_set_entry (upage, true, kpage);
  }
\end{verbatim}
\end{itemize}

\section{Implementation of Page and Frame Tables}\label{secimpl}
\subsection{Frame Table}
We have to implement two operations on the table of pages or frames.
\begin{enumerate}
\item to pick the last swapped-in entry to implement the LRU policy.
\item to search for an entry.
\end{enumerate}
The first operation is easy to implement for the list of entries; the
second, the hash of it. So we maintain the list and the hash of
entries at the same time. \texttt{init\_frame} is for the global
initialization function for frames.
\begin{verbatim}
- in frame.h
struct frame_table {
  uint8_t *page;
  struct thread *t;
  struct list_elem elem;
  struct hash_elem hash_elem;
}
struct list frame_list;
struct hash frame_hash;
- in init_frame, frame.c
  list_init(&frame_list);
  hash_init(&frame_hash, frame_hash_func, frame_less_func, NULL);
  lock_init(&lock_ft);
\end{verbatim}
The choice of hash function and order function for frames is
not unique. We hash a frame by the integral value of the page of the
frame and order frames by their page and their owner thread id.
\begin{verbatim}
- in frame.c
unsigned frame_hash_func ( const struct hash_elem *a, void *aux)
{ struct frame_table *pa = hash_entry (a, struct frame_table, hash_elem);
 return hash_bytes ( &pa->page, sizeof pa->page);
}
bool frame_less_func ( struct hash_elem *a, struct hash_elem *b, void *aux)
{ struct frame_table *pa = hash_entry ( a, struct frame_table, hash_elem);
 struct frame_table *pb = hash_entry ( b, struct frame_table, hash_elem);
 if ( pa->page == pb->page)
 {
  return pa->t->tid < pb->t->tid;
 }
 return pa->page < pb->page;
}
\end{verbatim}
Utility functions for maintaining the table is as follows.
\begin{verbatim}
- in make_frame_table, frame.c
  struct frame_table *fte = (struct frame_table *)malloc(sizeof(struct frame_table));
  ASSERT (fte != NULL);
  fte->page = page;
  fte->t = t;
  frame_table_insert(fte);
- in frame_table_insert, frame.c
 list_push_back (&frame_list,&f->elem);
 hash_insert (&frame_hash, &f->hash_elem);
- in frame_table_find, frame.c
  struct frame_table a;
  a.page = page;
  a.t = t;
  struct hash_elem *ret = hash_find ( &frame_hash, &a.hash_elem);
  if ( ret == NULL)
  {
    return NULL;
  }
  return hash_entry (ret, struct frame_table, hash_elem);
- in frame_table_free, frame.c
  struct list_elem *e;
  struct frame_table *f;
  for( e = list_begin(&frame_list); e != list_end(&frame_list); )
  {
    f = list_entry(e,struct frame_table,elem);      
    if (f->t == t)
    {
      e = list_remove (e);
      hash_delete ( &frame_hash, &f->hash_elem);
      free (f);
    }
    else
      e = list_next (e);
    }
  }
- in frame_table_entry_free, frame.c
  list_remove (&a->elem);
  hash_delete (&frame_hash, &a->hash_elem);
  free (a);
\end{verbatim}
If memory is scarce, we have to swap out an entry of frame
table. Following the LRU policy, \texttt{victim ()} victims the oldest
entry from the list of frame table entry and swap out it.
\begin{verbatim}
- in victim, frame.c
  fte = list_entry(list_pop_front(&frame_list), struct frame_table, elem);
  frame_thread = fte -> t;
  upage = fte -> page;
  /* Free FTE */
  hash_delete (&frame_hash, &fte->hash_elem);
  free(fte);
  p = page_table_find(frame_thread, upage);
  ASSERT (p != NULL);
  kpage = pagedir_get_page(frame_thread -> pagedir,upage);
  ASSERT (kpage != NULL);
  if ( pagedir_is_dirty ( frame_thread->pagedir, upage)) 
  {
    p->dirty = true;
  }
  pagedir_clear_page(frame_thread->pagedir, upage);\
  swap_out (p, kpage);
  return kpage;
\end{verbatim}

\subsection{Page Table}
The details are almost the same with that of frame tables.
\begin{verbatim}
- in page.h
struct page_table{
  bool swap;
  bool writable;
  uint8_t *page;
  uint8_t *kpage;
  int swp;
  bool dirty;
  bool is_loaded;
  off_t off;
  struct hash_elem hash_elem;
};
- in page.c
unsigned page_hash_func ( const struct hash_elem *a, void *aux UNUSED )
{
  struct page_table *pa = hash_entry ( a, struct page_table, hash_elem);
  return hash_bytes ( &pa->page, sizeof pa->page);
}
bool page_less_func ( struct hash_elem *a, struct hash_elem *b, void *aux)
{
  struct page_table *pa = hash_entry ( a, struct page_table, hash_elem);
  struct page_table *pb = hash_entry ( b, struct page_table, hash_elem);
  return pa->page < pb->page;
}
- in page.c
void page_table_init ( struct hash *pagetable )
{
  hash_init ( pagetable, page_hash_func, page_less_func, NULL);
}
void page_table_insert ( struct hash *pagetable, struct page_table *p)
{
  hash_insert ( pagetable, &p->hash_elem);
}
- in page_set_entry, page.c
  struct page_table *p = (struct page_table *)malloc(sizeof(struct page_table));
  ASSERT (p != NULL);
  p->swap = false;
  p->page = upage;
  p->writable = writable;
  p->kpage = kpage;
  p->is_loaded = false;
  p->dirty = false;
  page_table_insert ( &(thread_current()->pagetable), p);
- in page_table_find, page.c
  struct page_table a;
  a.page = page;
  struct hash_elem *ret = hash_find ( &t->pagetable, &a.hash_elem);
  if ( ret == NULL)
  {
    return NULL;
  }
  return hash_entry ( ret, struct page_table, hash_elem); 
- in page_table_entry_free, page.c
  struct thread *t = thread_current ();
  hash_delete ( &t->pagetable, &a->hash_elem);
  if ( a->swap )
  {
    d_use[ a->swp/8] = false;
  } 
  else if (a->is_loaded) 
  {
    pagedir_clear_page(t->pagedir, a->page);
  }
  free (a);
- in page_table_free
  hash_destroy ( &t->pagetable, page_destructor );
- in page_destructor, page.c
  struct page_table *pa = hash_entry ( a, struct page_table, hash_elem);
  if ( pa->swap )
  {
    d_use[ pa->swp/8] = false;
  }
  free ( pa );
\end{verbatim}

One think to focus on is \texttt{check\_fault}. It checks whether the
access to an address is valid. If it is valid, the function makes sure
that the corresponding page is swapped in.
\begin{enumerate}
\item If the page fault is due to an illegal access, terminate the
  process, freeing all the resources it acquired.
\begin{verbatim}
- in check_fault, page.c
  if (fault_addr >= PHYS_BASE || fault_addr == 0) {
    return true;
  }
  if (not_present) {
    if (fault_addr < DENY_ADDR) {
      return true;
    }
    if (pt_no(fault_pg) >= DENY_PAGE_NUM) {
      return true;
    }
  }
\end{verbatim}

\item Find the page for the accessed address.
\begin{verbatim}
- in check_fault, page.c
  pt = page_table_find (t, fault_pg);
  if (pt != NULL) {
    // When Page exists
    if (!lock_held_by_current_thread (&lock_ft)) 
    {
      get_lock_here = true;
      lock_acquire (&lock_ft);
    }
    if (pt->swap) {
      swap_in (pt, fault_pg);
      if (get_lock_here) 
        lock_release (&lock_ft);
      return false;
    }
    NOT_REACHED ();
  }
\end{verbatim}

If the page is not found, make a new page table entry.
\begin{verbatim}
- in check_fault, page.c
  if (fault_addr < PHYS_BASE - DENY_USER_MEM) {
    return true;
  }
  if (fault_addr <f->esp-32) {
    return true; 
  } 
  void *upage = fault_pg; 
  if (!lock_held_by_current_thread (&lock_ft)) 
  {
    get_lock_here = true;
    lock_acquire (&lock_ft);
  }
  kpage = palloc_get_page (PAL_USER | PAL_ZERO);
  if(kpage == NULL) 
    kpage = victim ();
  pagedir_set_page (t->pagedir, upage, kpage, true);
  make_frame_table (upage, t);
  page_set_entry (upage, true, kpage);
  if (get_lock_here)
  {
    lock_release (&lock_ft);
  }
  return false;
\end{verbatim}
\end{enumerate}

\section{Implementation of Swapping}\label{secswap}
Swapping in and out should read from and write to disk. To this, we
have to include \texttt{device/disk.[ch]}.

To swap out, we have to find an empty space.
\begin{verbatim}
- in find_empty, swap.c
for (i = 0; i < MAXDD; ++i)
{
  if (d_use[i] == false)
  {
    d_use[i] = true;
    return i * BYTE__;
  }
}
NOT_REACHED ();
\end{verbatim}
With the empty space found, save the location of the empty space to
the page table entry, and release the page.
\begin{verbatim}
- in swap_out, swap.c
pte->swp = sec_num;
pte->swap = true;
pte->kpage = NULL;
d = disk_get(1,1);
swap_d_write(d, sec_num, kpage);
\end{verbatim}
\texttt{swap\_d\_write} actually writes on the disk.
\begin{verbatim}
- in swap_d_write, swap.c
for (i=0;i<BYTE__;i++)
{
  disk_write(d,sec_num,page);
  sec_num++;
  page += P_SIZE;
}
\end{verbatim}

To swap in, we allocate a page. If allocation fails, we find a victim.
\begin{verbatim}
- in swap_in, swap.c
kpage = palloc_get_page (PAL_USER | PAL_ZERO);
if (kpage == NULL) 
  kpage = victim ();
\end{verbatim}
\texttt{swap\_d\_read} actually reads from the disk.
\begin{verbatim}
- in swap_d_read, swap.c
for(i=0;i<BYTE__;i++)
{
  disk_read(d,sec_num,page);
  sec_num++;
  page += P_SIZE;
}
\end{verbatim}
Finally, mark the disk space to be unused, and reorganize the page and
frame tables.
\begin{verbatim}
- in swap_in, swap.c
  d_use[pte->swp/8] = false;
  pte->swap = false;
  pte->kpage = kpage;
  make_frame_table ( addr, t );
  pagedir_set_page(t->pagedir, addr, kpage, true);
\end{verbatim}

\section{Conclusion}\label{secconclusion}
We implement an memory management system swapping memory frames
in-and-out which maintains the assignment of pages and
frames with an clever page fault handler. Collectively they are called
virtual memory. We pass all test cases except for
\texttt{pt-write-code}.

For \texttt{pt-write-code}, the program tries to write on code
section, which is prohibited. We checked that the page it wants to
write on is not writable at load time, but we could not figure out why
the program runs without an exception. We need the exact mechanism for
pintos to write on and read from memory to solve this problem.
\end{document}
