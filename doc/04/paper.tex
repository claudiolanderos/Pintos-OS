\documentclass[a4paper,article,11pt,oneside]{article}
\usepackage{fullpage}

\begin{document}
\title{Report for OS Project \#4 (2011 Fall)\\Team 071}
\author{Jeehoon Kang\\\texttt{jhkang@ropas.snu.ac.kr}\and
  Jonghyuk Lee\\\texttt{jonghyuk0605@gmail.com} \and
  Taekwon Lee\\\texttt{tkwonlee@hotmail.com} }
\maketitle

\section{Introduction}
For this project, we changed \texttt{filesys/cache.[ch],
  filesys/filesys.c, filesys/inode.c, threads/init.c,
  threads/thread.c, userprog/exception.c, userprog/process.c}.

This article is structured as follows. Section~\ref{seccache} shows
details of disk cache. TODO Section~\ref{secconclusion} concludes.

Note that our Mercurial repository is based at
\texttt{https://bitbucket.org/lunaticas/os}. Later, we address
revisions or files without explicitly mentioning the repository.

\section{Disk Cache}\label{seccache}
The naive implementation of Pintos distribution reads from and writes
to the disk inefficiently. To read from the disk, first read the
entire block which contains needed data from the disk, and copy the
data to the destination. To write to the disk, it first reads the entire
block which should be updated, updates the data, and writes the
updated block to the disk.

To overcome, we maintain a cache management system.
\begin{itemize}
\item A read or write request is divided into several block read and
  write requests.
\item Before read and write a block, a process should obtain
  appropriate privileges for the cache for the block.
\item Cache management system cares about concurrency synchronization.
\item Cache management system tries to reduce the number of disk
  operations by maintaining a pool of caches.
\item To fail-proof, Cache management system regularly applies the
  changes in the pool of caches to the disk.
\end{itemize}

This system is mainly implemented in \texttt{filesys/cache.[ch]}.

\subsection{Cache as a Resource}
Since disk cache is shared among processes, it should be treated as a
resources. We protect disk caches with readers-writer lock.
\begin{verbatim}
// cache.h
enum cache_lock_type
{
  READ_ONLY,
  READ_WRITE
};

// cache.c
struct cache_block 
{
  bool valid;
  bool loaded;
  bool dirty;
  disk_sector_t sector;
  uint8_t data[DISK_SECTOR_SIZE];
  int readers, writers, read_waiters, write_waiters;
  struct lock data_lock;
  struct lock rw_lock;
  struct condition read, write;
};
\end{verbatim}

\texttt{valid} means the cache is occupied. \texttt{loaded} means the
data is in sync with the disk. \texttt{dirty} means the data is
updated from the data in the disk. \texttt{sector} is the index for
the block in the disk. \texttt{data} is the block data. Other members
are for readers-writer lock.

\texttt{cache\_acquire, cache\_release} is for acquiring and releasing
the cache as a resource. The API is very standard and the
implementation follows conventional readers-writer lock scheme, so we
do not present a detailed description.

\subsection{Implementation of the Disk Cache}
The following function actually loads the data from the disk.
\begin{verbatim}
static void
cache_load (struct cache_block *cache)
{
  if (!cache->loaded)
  {
    disk_read (filesys_disk, cache->sector, cache->data);
    cache->loaded = true;
    cache->dirty = false;
  }
}
\end{verbatim}

The following function reads data from the disk cache. Before copying,
the function makes sure that the cache is loaded.
\begin{verbatim}
void
cache_read (struct cache_block *cache, int offset, int size, uint8_t *buffer)
{
  lock_acquire (&cache->data_lock);
  cache_load (cache);
  memcpy (buffer, cache->data + offset, size);
  lock_release (&cache->data_lock);
}
\end{verbatim}

The following function writes data to the disk cache. Before copying,
the function makes sure that the cache is loaded, unless the whole
cache is being written. The second function is only for convenience.
\begin{verbatim}
void
cache_write (struct cache_block *cache, int offset, int size, const uint8_t *buffer)
{
  lock_acquire (&cache->data_lock);
  if (offset == 0 && size == DISK_SECTOR_SIZE)
  {
    cache->loaded = true;
  }
  else
  {
    cache_load (cache);
  }
  memcpy (cache->data + offset, buffer, size);
  cache->dirty = true;
  lock_release (&cache->data_lock);
}

void
cache_fillzero (struct cache_block *cache)
{
  cache_load (cache);
  memset (cache->data, 0, sizeof (cache->data));
  cache->dirty = true;
}
\end{verbatim}

If we need to allocate a cache to a disk block, we need to find a
blank cache. The following function does it. At first, it finds a
non-occupied cache spot. If such cache spot does not exist, find a
cache which is not allocated by any processes.
\begin{verbatim}
static struct cache_block *
cache_victim (void)
{
  struct cache_block *cache;

  int count;
  for (count = 0; true; victim_idx = (victim_idx + 1) % CACHE_SIZE, ++count)
  {
    cache = &cache_table[victim_idx];

    if (!cache->valid)
    {
      break;
    }

    if (count >= CACHE_SIZE && cache->readers == 0 && cache->writers == 0 &&
        cache->read_waiters == 0 && cache->write_waiters == 0)
    {
      cache_flush (cache);
      break;
    }
  }

  cache->valid = false;
  cache->loaded = false;
  cache->dirty = false;

  return cache;
}
\end{verbatim}

The following function is for cache read-ahead. It pushes the
requested block to a designated list.
\begin{verbatim}
void
cache_read_request (disk_sector_t sector)
{
  struct read_info *read = (struct read_info *) malloc (sizeof (struct read_info));
  if (read == NULL)
  {
    return;
  }

  read->sector = sector;
  lock_acquire (&read_lock);
  list_push_back (&read_list, &read->elem);
  cond_signal (&read_list_nonempty, &read_lock);
  lock_release (&read_lock);
}
\end{verbatim}

For cache read-ahead, we maintain a daemon thread to work on the list.
\begin{verbatim}
// thread_start (void), threads/thread.c
#ifdef FILESYS
  thread_create ("cache_flushd", PRI_MIN, cache_flushd, NULL);
  thread_create ("cache_readd", PRI_MIN, cache_readd, NULL);
#endif

void
cache_readd (void *aux UNUSED)
{
  while (true)
  {
    lock_acquire (&read_lock);
    while (list_empty (&read_list))
    {
      cond_wait (&read_list_nonempty, &read_lock);
    }
    struct read_info *read = list_entry (list_pop_front (&read_list),
                                         struct read_info, elem);
    lock_release (&read_lock);

    struct cache_block *cache = cache_acquire (read->sector, READ_ONLY);
    cache_load (cache);
    cache_release (cache);
    free (read);
  }
}
\end{verbatim}

Another daemon thread is for regular disk flush to prevent accidental
disk failure. Note that \texttt{cache\_flush} is also used in
\texttt{cache\_victim} to clean the victim cache.
\begin{verbatim}
void
cache_flushd (void *aux UNUSED)
{
  while (true)
  {
    timer_sleep (5);
    cache_flush_all ();
  }
}

void cache_flush_all (void)
{
  int i;

  for (i = 0; i < CACHE_SIZE; ++i)
  {
    lock_acquire (&cache_lock);
    struct cache_block *cache = &cache_table[i];
    if (cache->valid)
    {
      cache_flush (cache);
    }
    lock_release (&cache_lock);  
  }
}

void
cache_flush (struct cache_block *cache)
{
  lock_acquire (&cache->data_lock);

  if (cache->loaded && cache->dirty)
  {
    disk_write (filesys_disk, cache->sector, cache->data);
    cache->dirty = false;
  }

  lock_release (&cache->data_lock);
}
\end{verbatim}

In addition, \texttt{filesys\_done} calls \texttt{cache\_flush\_all}
to ensure that the disk is in sync.

\subsection{API of Disk Cache}
Without disk caches, Pintos distribution called \texttt{disk\_read,
  disk\_write} for disk I/O. With disk caches, we first acquire
appropriate privileges for the cache for the block, I/O, then release
the cache.
\begin{verbatim}
- disk_read (filesys_disk, inode->sector, &inode->data);
+ struct cache_block *cache = cache_acquire (inode->sector, READ_ONLY);
+ cache_read (cache, 0, DISK_SECTOR_SIZE, (uint8_t *) &inode->data);
+ cache_release (cache);
\end{verbatim}

The changes are in \texttt{inode.c} and straightforward.

\section{Subdirectories}\label{secdir}
The file system implementation of Pintos distribution already includes
the notion of directories. However, to utilize these functions,
we have to bridge between the file system and the system calls. In
this section, we describes how to do it.

\subsection{Current Directory}
We store the current directory for each thread.
\begin{verbatim}
static bool
handle_chdir (const char *path)
{
  struct file *file = filesys_open (path);
  if (file == NULL)
  {
    return false;
  }

  thread_set_pwd (thread_current (), dir_open (file_get_inode (file)));
  file_close (file);
  return true;
}
\end{verbatim}
\begin{verbatim}
// init_thread, init.c
  #ifdef USERPROG
  t->pwd = NULL;
  #endif

void thread_set_pwd (struct thread *t, struct dir *dir)
{
  dir_close (t->pwd);
  t->pwd = dir;
}

const struct dir *thread_get_pwd (struct thread *t)
{
  return t->pwd;
}
\end{verbatim}

\subsection{Open}
System calls related to files only deals with file descriptors,
except for the \texttt{open} function. This function gets the path of
a file as an argument. We have to parse it to support subdirectories.

\begin{verbatim}
// syscall.c
static int
handle_open (const char *file_name)
{
  ...
  struct file *file_opened = filesys_open (file_name);
  ...
}

// filesys.c
struct file *
filesys_open (const char *path)
{
  struct inode *inode = dir_find (path);
  return file_open (inode);
}
\end{verbatim}

\texttt{dir\_find} finds the structure of directory for the given
path. It parses the path, iteratively finds the next-level
subdirectory, and returns the inode for the last element. Note that
the initial directory is the root directory or the current directory,
according to the first character of the path. The parsing is very
similar to that of argument, and the implementation is
straightforward.
\begin{verbatim}
// directory.h
struct inode *dir_find (const char *);
\end{verbatim}

\subsection{Create}
To create a file, we have to split the path to two parts: one for the
existing path, and the other for the name of the file to be
created. \texttt{dir\_split} do this.
\begin{verbatim}
// directory.c
bool dir_split (const char *path, struct inode **inode, char *to);
\end{verbatim}
It gets the path, and returns to the second and the third pointer
arguments. The second inode argument for the path where the new file
will be created, and the third string argument for the name of the
file to be created. Return value means whether parsing succeeded.

To create a file, first parse the path, locate the directory, allocate
a new inode for the file, then add it to the directory.
\begin{verbatim}
// syscall.c
static int
handle_create (const char *path, off_t file_size)
{
  if (!is_valid_memory_filecheck (path))
  {
    handle_exit (-1);
  }
  lock_acquire (&file_lock);
  int result = filesys_create (path, file_size);
  lock_release (&file_lock);
  return result;
}

// filesys.c
bool
filesys_create (const char *path, off_t initial_size) 
{
  disk_sector_t inode_sector = 0;
  struct inode *inode;
  struct dir *dir = NULL;
  char name[NAME_MAX + 1];

  bool success = (dir_split (path, &inode, name)
                  && (dir = dir_open (inode))
                  && free_map_allocate (1, &inode_sector)
                  && inode_create (inode_sector, initial_size, false)
                  && dir_add (dir, name, inode_sector));
  if (!success && inode_sector != 0) 
    free_map_release (inode_sector, 1);
  dir_close (dir);

  return success;
}
\end{verbatim}

\subsection{Remove}
To remove a file or directory, parse the path, and remove the file
from the directory.
\begin{verbatim}
bool
filesys_remove (const char *path)
{
  struct inode *inode;
  struct dir *dir = NULL;
  char name[NAME_MAX + 1];

  bool success = (dir_split (path, &inode, name)
                  && (dir = dir_open (inode))
                  && dir_remove (dir, name));
  dir_close (dir);

  return success;
}
\end{verbatim}

\subsection{Is it Directory?}
To discern directories from ordinary files, we have to store a flag
for each inode.
\begin{verbatim}
// inode.c
struct inode_disk
  {
    disk_sector_t start;                /* First data sector. */
    off_t length;                       /* File size in bytes. */
    bool is_dir;
    unsigned magic;                     /* Magic number. */
    uint32_t unused[124];               /* Not used. */
  };

bool inode_create (disk_sector_t sector, off_t length, bool is_dir)
{
  ...
}

bool
inode_isdir (const struct inode *inode)
{
  return inode->data.is_dir;
}

// directory.c
bool
dir_create (disk_sector_t sector, size_t entry_cnt) 
{
  return inode_create (sector, entry_cnt * sizeof (struct dir_entry), true);
}
\end{verbatim}

The flag is used to check whether a non-directory is opened as a
directory, and to dispatch the system calls across directories and
ordinary files.

To see whether a file descriptor represents a directory, just check
it.
\begin{verbatim}
static bool
handle_isdir (int fd)
{
  struct file *file = fd_get (fd);
  return inode_isdir (file_get_inode (file));
}
\end{verbatim}

Similarly for the inumber.
\begin{verbatim}
static int
handle_inumber (int fd)
{
  return inode_get_inumber (file_get_inode (fd_get (fd)));
}
\end{verbatim}

\subsection{Change Directory}
To change the current directory, open the directory, and set the
current directory of the current thread to be it.
\begin{verbatim}
static bool
handle_chdir (const char *path)
{
  struct file *file = filesys_open (path);
  if (file == NULL)
  {
    return false;
  }

  thread_set_pwd (thread_current (), dir_open (file_get_inode (file)));
  file_close (file);
  return true;
}
\end{verbatim}

\subsection{Make Directory}
Making a new directory is very similar to making a new file. The only
difference is calling \texttt{dir\_create} rather than
\texttt{inode\_create} and add \texttt{., ..} the the directory.
\begin{verbatim}
bool
filesys_mkdir (const char *path) 
{
  disk_sector_t inode_sector = 0;
  struct inode *inode;
  struct dir *dir = NULL;
  char name[NAME_MAX + 1];

  bool success = (dir_split (path, &inode, name)
                  && (dir = dir_open (inode))
                  && free_map_allocate (1, &inode_sector)
                  && dir_create (inode_sector, 16)
                  && dir_add (dir, name, inode_sector));
  dir_close (dir);
  success = success
                  && (dir = dir_open (inode_open (inode_sector)))
                  && dir_add (dir, ".", inode_sector)
                  && dir_add (dir, "..", inode_get_inumber (inode));
  if (!success && inode_sector != 0) 
    free_map_release (inode_sector, 1);
  dir_close (dir);

  return success;
}
\end{verbatim}

\subsection{Read Directory}
Though \texttt{dir\_readdir} is already implemented, we have to modify
this to prevent it from returning \texttt{., ..}.
\begin{verbatim}
bool
dir_readdir (struct dir *dir, char name[NAME_MAX + 1])
{
  struct dir_entry e;

  while (inode_read_at (dir->inode, &e, sizeof e, dir->pos) == sizeof e) 
    {
      dir->pos += sizeof e;
      if (e.in_use)
        {
            if (strcmp (e.name, ".") != 0 && strcmp (e.name, "..") != 0)
            {
              strlcpy (name, e.name, NAME_MAX + 1);
              return true;
            }
        } 
    }
  return false;
}
\end{verbatim}

Since \texttt{readdir} system call modifies the status of the file
descriptor and \texttt{dir\_readdir} modifies the status of
\texttt{dir} structure, we have to store a \texttt{dir} structure for
each file descriptor. We changed the file descriptor management system
to store and used it for \texttt{readdir} system call.
\begin{verbatim}
static bool
handle_readdir (int fd_id, char *name)
{
  if (!fd_is_valid (fd_id))
  {
    handle_exit (-1);
  }
  
  if (!is_valid_memory_filecheck (name))
  {
    handle_exit (-1);
  }
  
  struct dir *dir = fd_get_dir (fd_id);
  if (dir == NULL)
  {
    return false;
  }
  return dir_readdir (dir, name);
}
\end{verbatim}

\section{Conclusion}\label{secconclusion}
We implement a disk cache management system with easy API. It is
deliberately designed to avoid concurrency issues. We also implemented
a file system with subdirectories and corresponding system calls.
\end{document}
