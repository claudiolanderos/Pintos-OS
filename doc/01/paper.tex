\documentclass[a4paper,article,11pt,oneside]{article}
\usepackage{fullpage}

\begin{document}
\title{Report for OS Project \#1 (2011 Fall)\\Team 071}
\author{Jeehoon Kang\\\texttt{jhkang@ropas.snu.ac.kr}\and Jonghyuk Lee\\\texttt{jonghyuk0605@gmail.com} \and Taekwon Lee\\\texttt{tkwonlee@hotmail.com} }
\maketitle

This article is structured as follows. Section~\ref{secsleep} presents how to
make non-busy sleep function. Section~\ref{secpriority}
presents priority scheduler and section~\ref{secdonate} presents how
threads donate its priority when synchronous devices such as semaphore
and mutex are used. Section~\ref{secadv} presents 4.4BSD advanced
priority scheduler. Section~\ref{secconclusion} concludes.

Note that the Mercurial repository is based at
\texttt{https://bitbucket.org/lunaticas/os}. Later, we address
revisions or files whose repository is assumed.

We proudly report that we pass all 27 cases out of 27.
\begin{verbatim}
pass tests/threads/alarm-single
pass tests/threads/alarm-multiple
... (omitted)
pass tests/threads/mlfqs-nice-10
pass tests/threads/mlfqs-block
All 27 tests passed.
\end{verbatim}

\section{Non-busy Sleep Function}\label{secsleep}
Original implementation of \texttt{void
    timer\_sleep (int64\_t)} is as follows.
\begin{verbatim}
void
timer_sleep (int64_t ticks) 
{
  int64_t start = timer_ticks ();
  ASSERT (intr_get_level () == INTR_ON);
  while (timer_elapsed (start) < ticks) 
    thread_yield ();
}
\end{verbatim}
This is called busy-waiting implementation since the thread is not
excluded from the pool of running threads and get chances to run even
though it does not perform any significant computation. The goal of
this section is describe how to implement non-busy sleep.

To achieve this goal, the sleep function should depend on the
scheduler, since the sleeping threads should be excluded from the pool
of running threads. For this, we added a new thread status,
\texttt{THREAD\_SLEEP}.
\begin{verbatim}
(in threads/thread.h, enum thread_status)
     3.4      THREAD_RUNNING,     /* Running thread. */
     3.5      THREAD_READY,       /* Not running but ready to run. */
     3.6      THREAD_BLOCKED,     /* Waiting for an event to trigger. */
     3.7 +    THREAD_SLEEP,       /* Sleeping. */
     3.8      THREAD_DYING        /* About to be destroyed. */
     3.9    };
\end{verbatim}
When \texttt{void timer\_sleep (int64\_t)} is called, it calles
\texttt{void thread\_sleep (int64\_t)} in \texttt{thread.h}.
\begin{verbatim}
(in devices/timer.c)
     1.4  void
     1.5  timer_sleep (int64_t ticks) 
     1.6  {
     1.7 -  int64_t start = timer_ticks ();
     1.8 +  int64_t sleep_until = timer_ticks () + ticks;
     1.9  
    1.10    ASSERT (intr_get_level () == INTR_ON);
    1.11 -  while (timer_elapsed (start) < ticks) 
    1.12 -    thread_yield ();
    1.13 +  thread_sleep (sleep_until);
    1.14  }
\end{verbatim}
The role of \texttt{void thread\_sleep (int64\_t)} is exclude the
current thread from the pool of running threads,
i.e. \texttt{ready\_list}. The current thread is moved to
\texttt{sleep\_list}.
Every once a tick, \texttt{sleep\_list} is checked and threads which
should be awaken are moved to \texttt{ready\_list}, in the order of 1)
awakening time and 2) priority.
\begin{verbatim}
(in threads/thread.c, void thread_tick (void))
    2.82 +  /* Awake sleeping processes. */
    2.83 +  struct list_elem *e;
    2.84 +  struct thread *t;
    2.85 +  int64_t ticks = timer_ticks ();
    2.86 +
    2.87 +  while (!list_empty (&sleep_list))
    2.88 +  {
    2.89 +    e = list_min (&sleep_list, thread_sleep_less_func, NULL);
    2.90 +    t = list_entry (e, struct thread, elem);
    2.91 +
    2.92 +    if (t->sleep_until <= ticks)
    2.93 +    {
    2.94 +      t->status = THREAD_READY;
    2.95 +      list_remove (e);
    2.96 +      list_push_back (&ready_list, e);
    2.97 +    }
    2.98 +    else
    2.99 +    {
   2.100 +      break;
   2.101 +    }
   2.102 +  }
\end{verbatim}

To sum up, thanks to the update on the schuduler, the timer can
implement non-busy sleep function.

See revision 14.

\section{Priority Scheduler}\label{secpriority}

The basic behavior of priority scheduler is simple: the scheduler
always pick up the ready thread of the highest priority.
\begin{verbatim}
(in threads/thread.c, next_thread_to_run)
    1.88 -    return list_entry (list_pop_front (&ready_list), struct thread, elem);
    1.89 +  {
    1.90 +    struct list_elem *e = list_max (&ready_list,
    1.91 +                                    thread_priority_less_func,
    1.92 +                                    NULL);
    1.93 +    struct thread *t = list_entry (e, struct thread, elem);
    1.94 +    return t;
    1.95 +  }
    1.96  }
\end{verbatim}

However, to ensure that always the thread of the highest priority is
running, we have to consider three more cases: 1) thread creation, 2)
setting the priority of thread, 3) two or more threads are
simultaenuously waken up from sleep, and 4) two or more threads can be
unblocked from the same semaphore, lock, or conditional variable.

When the newly created thread has a higher priority than the current
one, the current thread immediately yields.
\begin{verbatim}
(in threads/thread.c, thread_create)
    1.22 +  if (thread_current ()->priority < priority)
    1.23 +  {
    1.24 +    thread_yield ();
    1.25 +  }
    1.26 +
\end{verbatim}

Setting the priority of the current thread may yield if the new
priority is not high enough.
\begin{verbatim}
(in threads/thread.c, thread_set_priority)
    1.59  void
    1.60  thread_set_priority (int new_priority) 
    1.61  {
    1.62 -  thread_current ()->priority = new_priority;
    1.63 +  struct thread *cur = thread_current ();
    1.64 +  int old_priority = cur->priority;
    1.65 +  enum intr_level old_level;
    1.66 +  
    1.67 +  ASSERT (!intr_context ());
    1.68 +
    1.69 +  old_level = intr_disable ();
    1.70 +  cur->priority = new_priority;
    1.71 +  if (new_priority < old_priority && !list_empty (&ready_list))
    1.72 +  {
    2.27 +    thread_yield ();
    1.79 +  }
    1.80 +  intr_set_level (old_level);  
    1.81  }
\end{verbatim}

When a thread is awaken, the current thread yields to ensure that the
thread of the highest priority is running. Also, when two threads wake
up at the same time, the one with a higher priority goes first.
\begin{verbatim}
(in threads/thread.c, thread_tick)
     1.4        t->status = THREAD_READY;
     1.5        list_remove (e);
     1.6        list_push_back (&ready_list, e);
     1.7 +      intr_yield_on_return ();
(in threads/thread.c, thread_sleep_less_func)
    1.36 -  return ta->sleep_until < tb->sleep_until;
    1.37 +  return (ta->sleep_until < tb->sleep_until) ||
    1.38 +      (ta->sleep_until == tb->sleep_until &&
    1.39 +       ta->priority > tb->priority);
\end{verbatim}

When a semaphore is up, the threads of the highest priority waiting
for the semaphore should wake up. Similarly for conditional variables,
too.

See revision 15, 19, 21, and 23.

\section{Priority Donation}\label{secdonate}

To implement priority donation, we have to make chains of locks and
donate priority through them. For semaphore and condition variable,
when they are released, the next thread to hold them is the one who
has the highest priority.

Threads have their own list of locks. Member \texttt{elem} of
\texttt{struck lock} and \texttt{waits\_for} of \texttt{struct thread}
are for it. Each lock has its highest priority in the lock chain it
belongs to. See \texttt{highest\_priority} of \texttt{struct lock}.

Since the priority of threads may changes according to donation, we
have to remember the explicitly set original priority. Member
\texttt{old\_priority} is for it.

\begin{verbatim}
(in threads/synch.h, struct lock)
     2.8 +    struct list_elem elem;      /* List element. */
     2.9 +    int highest_priority;       /* Highest priority of waiters. */
(in threads/thread.h, struct thread)
     4.6      int priority;                       /* Priority. */
     4.7 +    int old_priority;                   /* Base priority. */
    4.15 +    struct list locks;                  /* List of acquired locks. */
    4.16 +    struct lock *waits_for;             /* Lock waiting for acquire. */
\end{verbatim}

The initial value of \texttt{highest\_priority} of a lock is the
minimum priority. When the current thread wants to acquire a lock, it
donates the priority of the it to all the threads in the lock chain
whenever it is advantageous for them.

\begin{verbatim}
(in threads/synch.c, lock_init)
    1.25 +  lock->highest_priority = PRI_MIN;
(in threads/synch.c, lock_acquire)
  if(!thread_mlfqs)
  {
    enum intr_level old_level;

    old_level = intr_disable ();

      cur->waits_for = lock;

      struct lock *l;
      for (l = lock; l != NULL; l = l->holder->waits_for)
      {
        if (l->holder == NULL || l->highest_priority >= cur->priority) break;
        l->highest_priority = cur->priority;
        if (l->holder->priority < cur->priority) l->holder->priority = cur->priority;
      }
    intr_set_level (old_level);
  }

  sema_down (&lock->semaphore);
  lock->holder = cur;

  if(!thread_mlfqs)
  {
      cur->waits_for = NULL;
      list_push_back (&cur->locks, &lock->elem);
  }
\end{verbatim}

When the current thread releases a lock, the priority of current
thread is restored to the original priority before donation. 

\begin{verbatim}
(in threads/synch.c, lock_release)
    1.70    lock->holder = NULL;
    1.71 +  if(!thread_mlfqs)
    1.72 +  {
    1.73 +    lock->highest_priority = PRI_MIN;
    1.74 +    list_remove (&lock->elem);
    1.75 +  }
    1.76    sema_up (&lock->semaphore);
    1.77 +  if(!thread_mlfqs) thread_set_priority (thread_current ()->old_priority);
\end{verbatim}

In \texttt{set\_priority} of \texttt{thread.c}, with priority
donation, we have to consider priorities of locks acquired by the
current thread. So we find highest priority among acquired locks and
if it is greater than \texttt{new\_priority} then it becomes current
priority. Note that \texttt{old\_priority} has priority without
donation.

\begin{verbatim}
(in threads/thread.c, set_priority)
    3.46 +  enum intr_level old_level;
    3.47 +
    3.48 +  old_level = intr_disable ();
    3.49 +
    3.50    struct thread *cur = thread_current ();
    3.51    int old_priority = cur->priority;
    3.52 -  enum intr_level old_level;
    3.53  
    3.54 -  ASSERT (!intr_context ());
    3.55 +  cur->old_priority = new_priority;
    3.56  
    3.57 -  old_level = intr_disable ();
    3.58 +  struct list_elem *e;
    3.59 +  struct lock *t;
    3.60 +  for (e = list_begin (&cur->locks);
                 e != list_end (&cur->locks);
                 e = list_next (e))
    3.61 +  {
    3.62 +    t = list_entry (e, struct lock, elem);
    3.63 +    if (new_priority < t->highest_priority)
                new_priority = t->highest_priority;
    3.64 +  }
    3.65 +
    3.66    cur->priority = new_priority;
    3.67 -  if (cur->priority < old_priority && !list_empty (&ready_list))
    3.68 -  {
    3.69 +
    3.70 +  intr_set_level (old_level);
    3.71 +
    3.72 +  if (new_priority < old_priority)
    3.73      thread_yield ();
    3.74 -  }
    3.75 -  intr_set_level (old_level);
\end{verbatim}

See revision 21.

\section{4.4BSD Advanced Scheduler}\label{secadv}
The algorithm for this scheduler is presented in the Pintos
documentation, so we do not repeat it. This feature is activaated only
when \texttt{thread\_mlfqs} in \texttt{threads/thread.c} is true.

\subsection{Fixed Point Real Number Library}
We added a fixed point real number library in
\texttt{lib/kernel/real.[c,h]}. The description of structures and
functions are as follows.
\begin{itemize}
\item \texttt{struct real}\\
This structure is for a fixed point real number. As Pintos
documentation said, 17 bits for integer parts, 14 bits for fractional
parts, and 1 bit for sign constitute 32 bits of the integer.
\begin{verbatim}
(in lib/kernel/real.h)
struct real
{
  int bits;
}
\end{verbatim}

\item \texttt{struct real real\_of\_int (int)}\\
It returns the real number equivalent to the input integer number.

\item \texttt{int int\_of\_real (struct real)}\\
It returns the integer number equivalent to the input real number.

\item \texttt{struct real real\_add, real\_sub, real\_mult, real\_div
    (struct real, struct real)}\\
They are addition, subtraction, multiplication, and division for real
numbers.

\item \texttt{struct real real\_minus (struct real)}\\
This is the unary minus operator for real numbers.

\item \texttt{bool real\_less\_func (struct real, struct real)}\\
This is the comparison function for real numbers.

\end{itemize}

See revision 16 and 17.

\subsection{Niceness}
Niceness is added to a thread structure.
\begin{verbatim}
(in threads/thread.h, struct thread)
    6.24      char name[16];                      /* Name (for debugging purposes). */
    6.25      uint8_t *stack;                     /* Saved stack pointer. */
    6.26      int priority;                       /* Priority. */
    6.27 +    int nice;                           /* Niceness. */
\end{verbatim}
Now \texttt{init\_thread} accepts the niceness of the thread to be initialized.
\begin{verbatim}
(in threads/thread.c)
   5.269 -init_thread (struct thread *t, const char *name, int priority)
   5.270 +init_thread (struct thread *t, const char *name, int priority, int nice)
\end{verbatim}
With the above change, \texttt{thread\_create} passes the niceness of
the current thread to \texttt{init\_thread}.
\begin{verbatim}
(in threads/thread.c, thread_create)
   5.119 -  init_thread (t, name, priority);
   5.120 +  init_thread (t, name, priority, thread_current ()->nice);
\end{verbatim}

Setting the niceness of the current thread reschedule if necessary.
\begin{verbatim}
(in threads/thread.c, thread_set_nice)
    2.32    calculate_priority (thread_current (), NULL);
    2.33    if (cur->priority < old_priority && !list_empty (&ready_list))
    2.34    {
    2.41 +    thread_yield ();
    2.42    }
\end{verbatim}

See revision 16, 18, and 19.

\subsection{Average Load and Recent CPU}
The global variable \texttt{load\_avg} in \texttt{threads/thread.c} and
\texttt{recent\_cpu} of \texttt{struct thread} in
\texttt{threads/thread.h} is newly defined as real number.
\begin{verbatim}
(in threads/thread.c)
    5.15 +static struct real load_avg;
(in threads/thread.h, struct thread)
    6.28 +    struct real recent_cpu;             /* Recent CPU. */
    6.29      int64_t sleep_until;                /* For timer_sleep. */
    6.30      struct list_elem allelem;           /* List element for all threads list. */
\end{verbatim}

Every once a tick, \texttt{recent\_cpu} of the current thread is
added.
\begin{verbatim}
(in threads/thread.c, thread_tick)
    5.88 +    if (thread_current != idle_thread)
    5.89 +    {
    5.90 +      thread_current ()->recent_cpu = real_add (thread_current ()->recent_cpu,
    5.91 +                                                real_of_int (1));
    5.92 +    }
\end{verbatim}

Every once a second, \texttt{load\_avg} and \texttt{recent\_cpu} of each
thread is recalculated.
\begin{verbatim}
(in threads/thread.c, thread_tick)
    5.94 +    if (ticks % TIMER_FREQ == 0)
    5.95 +    {
    5.96 +      int ready_thread = list_size (&ready_list) + (thread_current () == idle_thread ? 0 : 1);
    5.97 +      load_avg = real_div (real_add (real_mult (real_of_int (59),
    5.98 +                                                load_avg),
    5.99 +                                     real_of_int (ready_thread)),
   5.100 +                           real_of_int (60));
   5.101 +      thread_foreach (calculate_recent_cpu,
   5.102 +                      NULL);
   5.103 +    }
\end{verbatim}

See revision 16.

\subsection{Priority}
Every once per 4 ticks, the priority of each thread is recalculated.
\begin{verbatim}
(in threads/thread.c, thread_tick)
   5.104 +    if (ticks % 4 == 0)
   5.105 +    {
   5.106 +      thread_foreach (calculate_priority,
   5.107 +                      NULL);
    1.16 +      if (!list_empty (&ready_list))
    1.17 +      {
    1.18 +        struct thread *t = next_thread_to_run ();
    1.19 +        if (thread_priority_less_func (&(cur->elem),
    1.20 +                                       &(t->elem),
    1.21 +                                       NULL)) {
    1.22 +          intr_yield_on_return ();
    1.23 +        }            
    1.24 +      }
   5.108 +    }
\end{verbatim}

See revision 16 and 18.

To sum up, priority of each thread is recalculated automatically with
recent CPU utilization and niceness of each thread, average load of
the CPU, etc. Updating the niceness reschedules if
necessary. Automation is due to \texttt{thread\_ticks}.

See revision 16, 17, 18, and 19.

\section{Conclusion}\label{secconclusion}
We made a non-busy sleep function, priority scheduler with donation,
and 4.4BSD advanced scheduler. We pass all test cases.

\end{document}
